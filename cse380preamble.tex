% This file can be included with a command like:
% % This file can be included with a command like:
% % This file can be included with a command like:
% % This file can be included with a command like:
% \input{\detokenize{~/.spacemacs.d/cse380preamble.tex}}

%
% Set lengths
%
\setlength{\oddsidemargin}{.25in}
\setlength{\evensidemargin}{.25in}
\setlength{\textwidth}{6in}
\setlength{\topmargin}{-0.4in}
\setlength{\textheight}{8.5in}

%
% mathify-- ensure argument is in math mode
%
\newcommand{\mathify}[1]{\ifmmode{#1}\else\mbox{$#1$}\fi}

%
% Header box to go at the top of the first page
%
\def\subjnum{CSE 380}
\def\subjname{Discrete Mathematics II}

\newcommand{\headerbox}[3]{
  \renewcommand{\thepage}{\arabic{page}}
  \noindent
  \begin{center}
    \framebox{
      \vbox{
        \hbox to 5.78in { {\bf \subjnum} \hfill {\bf \subjname} }
        \vspace{4mm}
        \hbox to 5.78in { {\Large \hfill #1  \hfill} }
        \vspace{2mm}
        \hbox to 5.78in { {\it #2 \hfill #3} }
        }
      }
  \end{center}
  \vspace*{4mm}
  }

%
% Some uses of headerbox
%
\newcommand{\placeholder}[1]{\headerbox{Placeholder}{Fall 2020}{#1}}
\newcommand{\assignment}[3]{\headerbox{#1}{Assignment #1}{#2}{#3}}
\newcommand{\quiz}[1]{\headerbox{#1 Quiz}{Name:}{}}
\newcommand{\handin}[3]{\headerbox{#1}{#2}{#3}}
\newcommand{\handout}[3]{\headerbox{#1}{Handout #2}{#3}}
\newcommand{\syllabus}{\headerbox{Syllabus}{Fall 2020}{Rick Neff}}

%
% Useful symbols
%
\newcommand{\qed}{\rule{7pt}{7pt}}
\newcommand{\ihat}{\hat{\imath}}
\newcommand{\jhat}{\hat{\jmath}}
\newcommand{\Nat}{\bf N}                        % natural numbers
\newcommand{\Int}{\mathbf{Z}}                   % integers
\newcommand{\Real}{\mathbf{R}}                  % reals
\newcommand{\Bool}{\it Bool}                    % booleans
\newcommand{\true}{\tt t}
\newcommand{\false}{\tt f}
\newcommand{\I}{\cal I}                         % interpretations
\newcommand{\M}{\cal M}                         % meaning functions
\newcommand{\A}{\cal A}                         % arithmetic interpretation
\newcommand{\B}{\cal B}                         % binary word interpretation
\newcommand{\union}{\cup}
\newcommand{\intersect}{\cap}
\newcommand{\Zp}{\mathbb{Z^{+}}}

%
% Useful functions
%
\newcommand{\abs}[1]{\mathify{\left| #1 \right|}}
\renewcommand{\Pr}[1]{\mathify{\mbox{Pr}\left[#1\right]}}
\newcommand{\Exp}[1]{\mathify{\mbox{Exp}\left[#1\right]}}
\newcommand{\set}[1]{\mathify{\left\{ #1 \right\}}}
\newcommand{\cset}[2]{\set{#1\ :\ #2}}  % a conditional notation to define sets
\newcommand{\lset}[2]{\set{#1,\ldots,#2}} % set {from,...,to}
\newcommand{\suchthat}{\vert}
\newcommand{\st}{\suchthat}
\newcommand{\forsome}{\exists}

%
% abbreviations
%
\newcommand{\ie}{{\em i.e.}}
\newcommand{\etc}{{\em etc.}}
\newcommand{\eg}{{\em e.g.}}
%\newcommand{\wlog}{\em w.l.o.g.}
\newcommand{\cf}{{\em cf.}}
\newcommand{\viz}{{\em viz.}}
\newcommand{\hint}{{\em Hint}:\ }              % for in-line hints
\newcommand{\note}{{\em Note}:\ }              % for in-line notes
\newcommand{\remark}{{\em Remark}\/:\ }        % for in-line remarks

%
% Use headerbox instead of \maketitle
% see cse380afterheader.tex
%
\renewcommand{\maketitle}{}

%
% Skip a line between paragraphs and don't indent them
%
\setlength\parindent{0pt}
\usepackage{parskip}

%
% For including python code
%
\usepackage{minted}
\usepackage{xcolor}
\definecolor{bg}{rgb}{0.9,0.9,0.9}

%
% Example (remove leading %s)
%
%\begin{minted}
%[
% frame=lines,framesep=2mm,baselinestretch=1.2,bgcolor=bg,fontsize=\footnotesize
%%linenos
%]
%{python}
%def f(m, n):
%    x = m + n
%    return (((x - 2) * (x - 1)) // 2) + m
%\end{minted}
}

%
% Set lengths
%
\setlength{\oddsidemargin}{.25in}
\setlength{\evensidemargin}{.25in}
\setlength{\textwidth}{6in}
\setlength{\topmargin}{-0.4in}
\setlength{\textheight}{8.5in}

%
% mathify-- ensure argument is in math mode
%
\newcommand{\mathify}[1]{\ifmmode{#1}\else\mbox{$#1$}\fi}

%
% Header box to go at the top of the first page
%
\def\subjnum{CSE 380}
\def\subjname{Discrete Mathematics II}

\newcommand{\headerbox}[3]{
  \renewcommand{\thepage}{\arabic{page}}
  \noindent
  \begin{center}
    \framebox{
      \vbox{
        \hbox to 5.78in { {\bf \subjnum} \hfill {\bf \subjname} }
        \vspace{4mm}
        \hbox to 5.78in { {\Large \hfill #1  \hfill} }
        \vspace{2mm}
        \hbox to 5.78in { {\it #2 \hfill #3} }
        }
      }
  \end{center}
  \vspace*{4mm}
  }

%
% Some uses of headerbox
%
\newcommand{\placeholder}[1]{\headerbox{Placeholder}{Fall 2020}{#1}}
\newcommand{\assignment}[3]{\headerbox{#1}{Assignment #1}{#2}{#3}}
\newcommand{\quiz}[1]{\headerbox{#1 Quiz}{Name:}{}}
\newcommand{\handin}[3]{\headerbox{#1}{#2}{#3}}
\newcommand{\handout}[3]{\headerbox{#1}{Handout #2}{#3}}
\newcommand{\syllabus}{\headerbox{Syllabus}{Fall 2020}{Rick Neff}}

%
% Useful symbols
%
\newcommand{\qed}{\rule{7pt}{7pt}}
\newcommand{\ihat}{\hat{\imath}}
\newcommand{\jhat}{\hat{\jmath}}
\newcommand{\Nat}{\bf N}                        % natural numbers
\newcommand{\Int}{\mathbf{Z}}                   % integers
\newcommand{\Real}{\mathbf{R}}                  % reals
\newcommand{\Bool}{\it Bool}                    % booleans
\newcommand{\true}{\tt t}
\newcommand{\false}{\tt f}
\newcommand{\I}{\cal I}                         % interpretations
\newcommand{\M}{\cal M}                         % meaning functions
\newcommand{\A}{\cal A}                         % arithmetic interpretation
\newcommand{\B}{\cal B}                         % binary word interpretation
\newcommand{\union}{\cup}
\newcommand{\intersect}{\cap}
\newcommand{\Zp}{\mathbb{Z^{+}}}

%
% Useful functions
%
\newcommand{\abs}[1]{\mathify{\left| #1 \right|}}
\renewcommand{\Pr}[1]{\mathify{\mbox{Pr}\left[#1\right]}}
\newcommand{\Exp}[1]{\mathify{\mbox{Exp}\left[#1\right]}}
\newcommand{\set}[1]{\mathify{\left\{ #1 \right\}}}
\newcommand{\cset}[2]{\set{#1\ :\ #2}}  % a conditional notation to define sets
\newcommand{\lset}[2]{\set{#1,\ldots,#2}} % set {from,...,to}
\newcommand{\suchthat}{\vert}
\newcommand{\st}{\suchthat}
\newcommand{\forsome}{\exists}

%
% abbreviations
%
\newcommand{\ie}{{\em i.e.}}
\newcommand{\etc}{{\em etc.}}
\newcommand{\eg}{{\em e.g.}}
%\newcommand{\wlog}{\em w.l.o.g.}
\newcommand{\cf}{{\em cf.}}
\newcommand{\viz}{{\em viz.}}
\newcommand{\hint}{{\em Hint}:\ }              % for in-line hints
\newcommand{\note}{{\em Note}:\ }              % for in-line notes
\newcommand{\remark}{{\em Remark}\/:\ }        % for in-line remarks

%
% Use headerbox instead of \maketitle
% see cse380afterheader.tex
%
\renewcommand{\maketitle}{}

%
% Skip a line between paragraphs and don't indent them
%
\setlength\parindent{0pt}
\usepackage{parskip}

%
% For including python code
%
\usepackage{minted}
\usepackage{xcolor}
\definecolor{bg}{rgb}{0.9,0.9,0.9}

%
% Example (remove leading %s)
%
%\begin{minted}
%[
% frame=lines,framesep=2mm,baselinestretch=1.2,bgcolor=bg,fontsize=\footnotesize
%%linenos
%]
%{python}
%def f(m, n):
%    x = m + n
%    return (((x - 2) * (x - 1)) // 2) + m
%\end{minted}
}

%
% Set lengths
%
\setlength{\oddsidemargin}{.25in}
\setlength{\evensidemargin}{.25in}
\setlength{\textwidth}{6in}
\setlength{\topmargin}{-0.4in}
\setlength{\textheight}{8.5in}

%
% mathify-- ensure argument is in math mode
%
\newcommand{\mathify}[1]{\ifmmode{#1}\else\mbox{$#1$}\fi}

%
% Header box to go at the top of the first page
%
\def\subjnum{CSE 380}
\def\subjname{Discrete Mathematics II}

\newcommand{\headerbox}[3]{
  \renewcommand{\thepage}{\arabic{page}}
  \noindent
  \begin{center}
    \framebox{
      \vbox{
        \hbox to 5.78in { {\bf \subjnum} \hfill {\bf \subjname} }
        \vspace{4mm}
        \hbox to 5.78in { {\Large \hfill #1  \hfill} }
        \vspace{2mm}
        \hbox to 5.78in { {\it #2 \hfill #3} }
        }
      }
  \end{center}
  \vspace*{4mm}
  }

%
% Some uses of headerbox
%
\newcommand{\placeholder}[1]{\headerbox{Placeholder}{Fall 2020}{#1}}
\newcommand{\assignment}[3]{\headerbox{#1}{Assignment #1}{#2}{#3}}
\newcommand{\quiz}[1]{\headerbox{#1 Quiz}{Name:}{}}
\newcommand{\handin}[3]{\headerbox{#1}{#2}{#3}}
\newcommand{\handout}[3]{\headerbox{#1}{Handout #2}{#3}}
\newcommand{\syllabus}{\headerbox{Syllabus}{Fall 2020}{Rick Neff}}

%
% Useful symbols
%
\newcommand{\qed}{\rule{7pt}{7pt}}
\newcommand{\ihat}{\hat{\imath}}
\newcommand{\jhat}{\hat{\jmath}}
\newcommand{\Nat}{\bf N}                        % natural numbers
\newcommand{\Int}{\mathbf{Z}}                   % integers
\newcommand{\Real}{\mathbf{R}}                  % reals
\newcommand{\Bool}{\it Bool}                    % booleans
\newcommand{\true}{\tt t}
\newcommand{\false}{\tt f}
\newcommand{\I}{\cal I}                         % interpretations
\newcommand{\M}{\cal M}                         % meaning functions
\newcommand{\A}{\cal A}                         % arithmetic interpretation
\newcommand{\B}{\cal B}                         % binary word interpretation
\newcommand{\union}{\cup}
\newcommand{\intersect}{\cap}
\newcommand{\Zp}{\mathbb{Z^{+}}}

%
% Useful functions
%
\newcommand{\abs}[1]{\mathify{\left| #1 \right|}}
\renewcommand{\Pr}[1]{\mathify{\mbox{Pr}\left[#1\right]}}
\newcommand{\Exp}[1]{\mathify{\mbox{Exp}\left[#1\right]}}
\newcommand{\set}[1]{\mathify{\left\{ #1 \right\}}}
\newcommand{\cset}[2]{\set{#1\ :\ #2}}  % a conditional notation to define sets
\newcommand{\lset}[2]{\set{#1,\ldots,#2}} % set {from,...,to}
\newcommand{\suchthat}{\vert}
\newcommand{\st}{\suchthat}
\newcommand{\forsome}{\exists}

%
% abbreviations
%
\newcommand{\ie}{{\em i.e.}}
\newcommand{\etc}{{\em etc.}}
\newcommand{\eg}{{\em e.g.}}
%\newcommand{\wlog}{\em w.l.o.g.}
\newcommand{\cf}{{\em cf.}}
\newcommand{\viz}{{\em viz.}}
\newcommand{\hint}{{\em Hint}:\ }              % for in-line hints
\newcommand{\note}{{\em Note}:\ }              % for in-line notes
\newcommand{\remark}{{\em Remark}\/:\ }        % for in-line remarks

%
% Use headerbox instead of \maketitle
% see cse380afterheader.tex
%
\renewcommand{\maketitle}{}

%
% Skip a line between paragraphs and don't indent them
%
\setlength\parindent{0pt}
\usepackage{parskip}

%
% For including python code
%
\usepackage{minted}
\usepackage{xcolor}
\definecolor{bg}{rgb}{0.9,0.9,0.9}

%
% Example (remove leading %s)
%
%\begin{minted}
%[
% frame=lines,framesep=2mm,baselinestretch=1.2,bgcolor=bg,fontsize=\footnotesize
%%linenos
%]
%{python}
%def f(m, n):
%    x = m + n
%    return (((x - 2) * (x - 1)) // 2) + m
%\end{minted}
}

%
% Set lengths
%
\setlength{\oddsidemargin}{.25in}
\setlength{\evensidemargin}{.25in}
\setlength{\textwidth}{6in}
\setlength{\topmargin}{-0.4in}
\setlength{\textheight}{8.5in}

%
% mathify-- ensure argument is in math mode
%
\newcommand{\mathify}[1]{\ifmmode{#1}\else\mbox{$#1$}\fi}

%
% Header box to go at the top of the first page
%
\def\subjnum{CSE 380}
\def\subjname{Discrete Mathematics II}

\newcommand{\headerbox}[3]{
  \renewcommand{\thepage}{\arabic{page}}
  \noindent
  \begin{center}
    \framebox{
      \vbox{
        \hbox to 5.78in { {\bf \subjnum} \hfill {\bf \subjname} }
        \vspace{4mm}
        \hbox to 5.78in { {\Large \hfill #1  \hfill} }
        \vspace{2mm}
        \hbox to 5.78in { {\it #2 \hfill #3} }
        }
      }
  \end{center}
  \vspace*{4mm}
  }

%
% Some uses of headerbox
%
\newcommand{\placeholder}[1]{\headerbox{Placeholder}{Fall 2020}{#1}}
\newcommand{\assignment}[3]{\headerbox{#1}{Assignment #1}{#2}{#3}}
\newcommand{\quiz}[1]{\headerbox{#1 Quiz}{Name:}{}}
\newcommand{\handin}[3]{\headerbox{#1}{#2}{#3}}
\newcommand{\handout}[3]{\headerbox{#1}{Handout #2}{#3}}
\newcommand{\syllabus}{\headerbox{Syllabus}{Fall 2020}{Rick Neff}}

%
% Useful symbols
%
\newcommand{\qed}{\rule{7pt}{7pt}}
\newcommand{\ihat}{\hat{\imath}}
\newcommand{\jhat}{\hat{\jmath}}
\newcommand{\Nat}{\bf N}                        % natural numbers
\newcommand{\Int}{\mathbf{Z}}                   % integers
\newcommand{\Real}{\mathbf{R}}                  % reals
\newcommand{\Bool}{\it Bool}                    % booleans
\newcommand{\true}{\tt t}
\newcommand{\false}{\tt f}
\newcommand{\I}{\cal I}                         % interpretations
\newcommand{\M}{\cal M}                         % meaning functions
\newcommand{\A}{\cal A}                         % arithmetic interpretation
\newcommand{\B}{\cal B}                         % binary word interpretation
\newcommand{\union}{\cup}
\newcommand{\intersect}{\cap}
\newcommand{\Zp}{\mathbb{Z^{+}}}

%
% Useful functions
%
\newcommand{\abs}[1]{\mathify{\left| #1 \right|}}
\renewcommand{\Pr}[1]{\mathify{\mbox{Pr}\left[#1\right]}}
\newcommand{\Exp}[1]{\mathify{\mbox{Exp}\left[#1\right]}}
\newcommand{\set}[1]{\mathify{\left\{ #1 \right\}}}
\newcommand{\cset}[2]{\set{#1\ :\ #2}}  % a conditional notation to define sets
\newcommand{\lset}[2]{\set{#1,\ldots,#2}} % set {from,...,to}
\newcommand{\suchthat}{\vert}
\newcommand{\st}{\suchthat}
\newcommand{\forsome}{\exists}

%
% abbreviations
%
\newcommand{\ie}{{\em i.e.}}
\newcommand{\etc}{{\em etc.}}
\newcommand{\eg}{{\em e.g.}}
%\newcommand{\wlog}{\em w.l.o.g.}
\newcommand{\cf}{{\em cf.}}
\newcommand{\viz}{{\em viz.}}
\newcommand{\hint}{{\em Hint}:\ }              % for in-line hints
\newcommand{\note}{{\em Note}:\ }              % for in-line notes
\newcommand{\remark}{{\em Remark}\/:\ }        % for in-line remarks

%
% Use headerbox instead of \maketitle
% see cse380afterheader.tex
%
\renewcommand{\maketitle}{}

%
% Skip a line between paragraphs and don't indent them
%
\setlength\parindent{0pt}
\usepackage{parskip}

%
% For including python code
%
\usepackage{minted}
\usepackage{xcolor}
\definecolor{bg}{rgb}{0.9,0.9,0.9}

%
% Example (remove leading %s)
%
%\begin{minted}
%[
% frame=lines,framesep=2mm,baselinestretch=1.2,bgcolor=bg,fontsize=\footnotesize
%%linenos
%]
%{python}
%def f(m, n):
%    x = m + n
%    return (((x - 2) * (x - 1)) // 2) + m
%\end{minted}
